%\documentclass{dgleich-article}
%\RequirePackage[l2tabu, orthodox]{nag}
\documentclass[longtitle,sidenotes,nofancyfonts]{dgleich-article}
%\documentclass{article}
%\documentclass{article}
%\usepackage{showframe}
\usepackage{dgleich-math}



\title{Shining a light into the digital dark age with computational approaches to digital stewardship}
\author{David F. Gleich, Vinayak Ganeshan, Ying Wang, Nathan Sakunkoo, Farnaz Ronaghi, Xiangrui Meng, Amin Saberi, Margot Gerritsen}

\begin{document}
%\rule{\linewidth}
%Test 
%\hrule

 \maketitle

\begin{abstract}
 This lecture covers eigenvalue algorithms.
\end{abstract}

\section{Review}
Last lecture, we saw...\footnote{Blah}

\section{The Lanczos algorithm}
Given a matrix $A$, we derive the Lanczos algorithm by asking:
is there any way to \emph{incrementally} construct a reduction
of a matrix?
Recall that ``direct'' eigenvalue methods begin by reducing
the matrix to a simpler form -- tridiagonal form for a
symmetric matrix -- before applying the QR iteration.

\begin{theorem}\marginnote{\sffamily THEOREM \thetheorem\\[1ex]}%\marginnote{Test}
 Test
\end{theorem}

\section{Math test}
Check that math-blackboard-bold works:
Let $\vx \in \mathbb{R}^n$.

Here, we look at all the norm commands
\[
 \absof{\frac{1}{2}} \qquad \absof*{\frac{1}{2}} \qquad \absof[\Bigg]{\frac{1}{2}}
\]

\[
 \normof{\frac{1}{2}} \qquad \normof*{\frac{1}{2}}
 \qquad \normof[1]{\frac{1}{2}} 
 \qquad \normof[1][\Bigg]{\frac{1}{2}} 
\]

\[ \Eof{\vx(A)} \]
\[ \hugeEof{\vx(A)} \]
\[ \hugeprob{\vx(A)} \hugeprob[x][\Bigg]{e^{-x^2}} \]

\section{Fonts}

\subsection{Typewriter}

With the \verb#fancyfonts# option, 
we use the \verb#beramono# font -- a nicely crafted typewriter font that contrasts
with the fine serifs of Adobe's Minion Pro and has a complementary softness
to the starkness of Helvetica.  

\begin{quote}
\ttfamily 
Here is an example of typewriter text.
\end{quote}

Without the fancyfonts option, we'd like to use the standard computer modern
typewriter typeface.  However, computer modern lacks a native bold typewriter type.  
\begin{quote}
 \ttfamily
\textbf{Here} is an example of \textbf{BOLD} typewriter text.
\end{quote}

Here is an example of algorithmic pseudocode
\begin{pseudocode}
initialize $\mB_0$, and $k = 0$
for $k=0, ...$ and while $\vx_k$ does not satisfy the conditions we want ...
  solve for the search direction $\mB_k \vp_k = -\vg$
  compute a line search $\alpha_k$
  update $\vx_{k+1} = \vx_{k} + \alpha \vp_k$
  update $\mB_{k+1}$ based on $\vx_{k+1}$
\end{pseudocode}

\end{document}
