% language
\usepackage[english]{babel}

% setup fonts
%\usepackage{GillSans}
\usepackage[scaled=0.85]{helvet}
\usefonttheme{professionalfonts}
\usepackage[scaled=0.8]{beramono}
\usepackage{arev}
\usepackage{fix-cm}
\usepackage[T1]{fontenc}

% packages
\usepackage{varwidth}
\usepackage{algorithmic}
\usepackage{rotating}
\let\Tiny=\tiny
\usepackage[beamer,matrixmath,matrixshorts,vectorshorts,statisticsmath,genericmath,tkzberge]{dgleich}
\graphicspath{{./}{./figures/}{./code/}}
\usepackage{microtype}
\usepackage{textcase}
\usepackage{tabularx}
\usepackage[absolute,overlay]{textpos}
\usepackage[texcoord]{eso-pic}



% display agnostic commands

% use a fancysans 
% \newcommand{\fancysansfamily}{GillSans-TLF}
% \newcommand{\fancysans}{\usefont{T1}{GillSans-TLF}{m}{n}}
% \newcommand{\fancysanstext}[1]{{\fancysans #1}}
% \newcommand{\fancysanstextit}[1]{{\usefont{T1}{GillSans-TLF}{m}{it} #1}}

% use Helvetica
\newcommand{\fancysansfamily}{phv}
\newcommand{\fancysans}{\usefont{T1}{phv}{m}{n}}
\newcommand{\fancysanstext}[1]{{\fancysans #1}}
\newcommand{\fancysanstextit}[1]{{\usefont{T1}{GillSans-TLF}{m}{it} #1}}

% define a command for fancy text fractions if the font supports it
\newcommand{\textfrac}[1]{#1}

% change the margins
\newenvironment{changemargin}[2]{%
\begin{list}{}{%
\setlength{\topsep}{0pt}%
\setlength{\leftmargin}{#1}%
\setlength{\rightmargin}{#2}%
\setlength{\listparindent}{\parindent}%
\setlength{\itemindent}{\parindent}%
\setlength{\parsep}{\parskip}%
}%
\item[]}{\end{list}}

%adjust the TPHorizModule and TPHorizModule units to the displayed mm %grid
%\TPGrid[0mm,0mm]{100}{100}

%puts a graphic at the absolute position described by the grid
%#1 x, #2 y, #3 , #4 graphic
\newcommand\putpic[4][]{%
        \begin{textblock}{0}(#2,#3)%
  \begin{picture}(0,0)(0,0)\includegraphics[#1]{#4}\end{picture}%
     \end{textblock}%
} 
\newcommand{\putat}[3]{%
\begin{textblock}{0}(#1,#2)%
\begin{picture}(0,0)(0,0){#3}\end{picture}%
\end{textblock}}%




%%
%% SETUP the slide appearance
%% These commands are often nasty, we begin by defining a set of colors
%% then the next include actually specifies all the beamer layout commands.
%%

% Colors
\definecolor{accentlight}     {RGB}{31,73,125}
\definecolor{accentheavy}      {RGB}{153,0,0}
\definecolor{shadelight}{RGB}{142,180,227} % light blue
%\definecolor{shadeheavy}{RGB}{203,203,203}
\definecolor{shadeheavy}{RGB}{196,189,151}
\colorlet{shadelight}{shadeheavy!50!white}
\definecolor{backgroundaccent}{RGB}{196,189,151}
\definecolor{background}{RGB}{238,236,225}
\definecolor{neutralshade}{gray}{0.85}  

% derived colors
\colorlet{shadecolor}{neutralshade}
\colorlet{lighttextcolor}{shadecolor!50!black}
\colorlet{boxshade}{backgroundaccent}
\colorlet{boxframe}{backgroundaccent}

\usetheme{GleichPosterPaloAlto}

% setup Poster
\usepackage[orientation=landscape,size=a0,scale=1.1]{beamerposter}       

% setup pieces on top of the poster.


% Poster columns
% fix spacing in the columns
\makeatletter
\newcommand{\defaultmetrics}{%
% Setup parskip (just ripped from parskip.sty)
\parskip=0.5\baselineskip \advance\parskip by 0pt plus 2pt%
\parindent=\z@%
\def\@listI{\leftmargin\leftmargini%
   \partopsep\z@% added by G. Partosch
   \topsep\z@ \parsep\parskip \itemsep\z@}%
\let\@listi\@listI%
\@listi%
\def\@listii{\leftmargin\leftmarginii%
   \labelwidth\leftmarginii\advance\labelwidth-\labelsep%
   \partopsep\z@% added by G. Partosch
   \topsep\z@ \parsep\parskip \itemsep\z@}%
\def\@listiii{\leftmargin\leftmarginiii%
    \labelwidth\leftmarginiii\advance\labelwidth-\labelsep%
    \partopsep\z@% added by G. Partosch
    \topsep\z@ \parsep\parskip \itemsep\z@}%
\justifying% Make everything Justified    
}
\newcommand{\postercolumn}[2][t]{%
  \column[#1]{#2}%
  \normalsize\defaultmetrics\ignorespaces%
}
\makeatother


%% 
%% Display specific commands.  
%%

\newcommand{\reftext}[1]{\fancysanstextit{\textcolor{accentlight}{#1}}}

\newcommand{\noterlap}[1]{\rlap{\color{lighttextcolor}#1}}

\newcommand{\plotbox}[1]{%
  \setlength\fboxrule{1pt}\fcolorbox{boxframe}{boxshade}{#1}%
}  
\newcommand{\cplotbox}[1]{%
  \begin{center}\setlength\fboxrule{1pt}\fcolorbox{boxframe}{boxshade}{#1}\end{center}%
}  
\newcommand{\mathbox}[2][0.8]{%
  \begin{center}%
    \setlength\fboxrule{1pt}\fcolorbox{boxframe}{boxshade}{%
    \begin{beamercolorbox}[sep=1em,wd=#1\linewidth]{mybox}%
    #2%
    \end{beamercolorbox}%
    }%
  \end{center}%
}

\newcommand{\deemph}[1]{%
  {\itshape\color{lighttextcolor}%
  #1%
 }
}

\newcommand{\captext}[1]{%
 {\scriptsize\deemph{#1}}%
}


\newenvironment{highlight}[1][1ex]{%
  \begin{list}{}{%
    \setlength{\topsep}{1ex}%
    \setlength{\itemsep}{1ex}%
    %\setlength{\parsep}{2pt}%
    \setlength{\listparindent}{0pt}%
    \setlength{\itemindent}{0pt}%
    \setlength{\leftmargin}{1ex}%
    \setlength{\rightmargin}{0pt}%
  }\color{accentlight}\item%
}{%  
  \end{list}%
}

